% !TeX spellcheck = en_GB
\ifcsname SlidesDistr\endcsname%
	\documentclass[handout,aspectratio=169]{beamer}
\else%
	\documentclass[aspectratio=169]{beamer}
\fi%
\usepackage{fontspec}
\usepackage[T1]{fontenc}
\usepackage{amsmath}
\usepackage{amsfonts}
\usepackage{amssymb}
\usepackage{graphicx}
\usepackage{csquotes}
\usepackage{booktabs}
\usepackage{multicol}
\usepackage{enumerate}
\usepackage{microtype}
\usepackage[labelfont=bf,font={small}]{caption}
\usepackage{hyperref}
\usepackage{booktabs}
\usepackage{subcaption}
\usepackage{fancyhdr}
\usepackage{pdfpages}
\usepackage{siunitx}
\usepackage{tikz}
\usepackage{mdframed}

\defaultfontfeatures{Mapping=tex-text}
\newfontfamily\symbolfont{Symbola}
\newfontfamily\quotefont{GentiumPlus}

\usepackage[sorting=none]{biblatex}
\addbibresource{../bibliography.bib}

\author{Chris Eliasmith}


\renewcommand{\vec}[1]{{\mathbf{#1}}}
\newcommand{\mat}[1]{{\mathbf{#1}}}
\newcommand{\T}{\ensuremath{\mathsf{T}}}
\renewcommand{\epsilon}{\varepsilon}
\renewcommand{\phi}{\varphi}

% Tango color palette
\definecolor{butter1}{HTML}{FCE94F}
\definecolor{butter2}{HTML}{EDD400}
\definecolor{butter3}{HTML}{C4A000}
\definecolor{orange1}{HTML}{FCAF3E}
\definecolor{orange2}{HTML}{F57900}
\definecolor{orange3}{HTML}{CE5C00}
\definecolor{chocolate1}{HTML}{E9B96E}
\definecolor{chocolate2}{HTML}{C17D11}
\definecolor{chocolate3}{HTML}{8F5902}
\definecolor{chameleon1}{HTML}{8AE234}
\definecolor{chameleon2}{HTML}{73D216}
\definecolor{chameleon3}{HTML}{4E9A06}
\definecolor{skyblue1}{HTML}{729FCF}
\definecolor{skyblue2}{HTML}{3465A4}
\definecolor{skyblue3}{HTML}{204A87}
\definecolor{plum1}{HTML}{AD7FA8}
\definecolor{plum2}{HTML}{75507B}
\definecolor{plum3}{HTML}{5C3566}
\definecolor{scarletred1}{HTML}{EF2929}
\definecolor{scarletred2}{HTML}{CC0000}
\definecolor{scarletred3}{HTML}{A40000}
\definecolor{aluminium1}{HTML}{EEEEEC}
\definecolor{aluminium2}{HTML}{D3D7CF}
\definecolor{aluminium3}{HTML}{BABDB6}
\definecolor{aluminium4}{HTML}{888A85}
\definecolor{aluminium5}{HTML}{555753}
\definecolor{aluminium6}{HTML}{2E3436}

\definecolor{violet}{HTML}{AA305C}
\definecolor{uwyellow}{HTML}{FDD433}
\definecolor{background}{HTML}{F9F9F6}
\definecolor{text}{HTML}{000000}

\definecolor{uweng1}{HTML}{D1B2EE}
\definecolor{uweng2}{HTML}{BF33DE}
\definecolor{uweng3}{HTML}{8001B3}
\definecolor{uweng4}{HTML}{56048A}

\setbeamercolor{title}{fg=violet}
\setbeamercolor{frametitle}{fg=black}
\setbeamercolor{structure}{fg=aluminium5}
\setbeamercolor{normal text}{fg=text}

\setbeamertemplate{navigation symbols}{}
\setbeamertemplate{footline}[frame number]

\hypersetup{%
	colorlinks=false,% hyperlinks will be black
	urlbordercolor=aluminium4,% hyperlink borders will be red
	pdfborderstyle={/S/U/W 0.5}% border style will be underline of width 1pt
}

\makeatletter
\newcommand{\superimpose}[2]{%
	{\ooalign{{#1}\hidewidth\cr{#2}\hidewidth\cr}}}
\makeatother
\newcommand{\SolidCircle}[2]{\superimpose{\color{#1}\symbolfont ⬤}{\textbf{\color{white}#2}}\hspace{1em}}
\newcommand{\OPlus}{\SolidCircle{chameleon3}{\kern0.75pt+}}
\newcommand{\OMeh}{\SolidCircle{uwyellow}{~}}
\newcommand{\OMinus}{\SolidCircle{scarletred3}{\kern2.25pt--}}

\newcommand{\hl}[1]{\colorbox{uwyellow}{{\color{black}\textbf{#1}}}}

\newcommand{\ImageSources}[1]{%
	\begin{columns}%
		\column{1.1\textwidth}%
		\raggedright%
		\tiny\color{aluminium4}%
		\setlength\lineskip{1em}%
		\textbf{Image Sources.}	{#1}%
	\end{columns}}

\newcommand{\ColorRect}[3]{{\color{#1}\rule{#2}{#3}}}
\setbeamertemplate{headline}{\ColorRect{black}{\textwidth}{4pt}\newline\ColorRect{uweng1}{0.25\textwidth}{4pt}\ColorRect{uweng2}{0.25\textwidth}{4pt}\ColorRect{uweng3}{0.25\textwidth}{4pt}\ColorRect{uweng4}{0.25\textwidth}{4pt}}

\newcommand{\MakeTitle}{%
	\vspace{0.5cm}%
	{\textbf{\inserttitle}}\\[0.5cm]%
	\insertauthor\\[0.5cm]%
	\insertdate\\%
	\begin{itemize}
		\item Slide design: Andreas Stöckel
		\item Content: Terry Stewart, Andreas Stöckel, Chris Eliasmith
	\end{itemize}
 	\includegraphics[width=7cm]{../assets/uwlogo_eng.pdf}%
}

\newcommand{\handwritingframe}{%
	\begin{frame}
		\begin{columns}
			\column{\paperwidth}
			\includegraphics{../assets/handwriting_lines.pdf}
		\end{columns}
	\end{frame}	
}

\newcommand{\imageframe}[1]{%
	\setbeamertemplate{navigation symbols}{}%
	\begin{frame}[plain,noframenumbering]%
		\begin{tikzpicture}[remember picture,overlay]%
		\node[at=(current page.center)] {%
			\includegraphics[width=\paperwidth]{#1}%
		};%
		\end{tikzpicture}%
	\end{frame}%
}

\newcommand{\videoframe}[3][mp4]{%
	\begin{frame}[plain,noframenumbering]%
		\hypersetup{%
			pdfborderstyle={/S/U/W 0}% border style will be underline of width 1pt
		}%
		\begin{tikzpicture}[remember picture,overlay]%
		\node[at=(current page.center)] {%
			\includegraphics[width=\paperwidth]{{video/#2_#3}.jpg}%
		};%
		\node[at=(current page.center)] {%
			\ifcsname SlidesDistr\endcsname%
				\href{https://youtu.be/#3}{\includegraphics[width=2cm]{../assets/play_button.pdf}}%
			\else%
				\href{video/#2_#3.#1}{\includegraphics[width=2cm]{../assets/play_button.pdf}}%
			\fi%
		};%
		\end{tikzpicture}%
	\end{frame}%
}

\newcommand{\includevideo}[4][mp4]{%
	\begingroup%
	\hypersetup{%
		pdfborderstyle={/S/U/W 0}% border style will be underline of width 1pt
	}%
	\begin{tikzpicture}%
	\node (A) {%
		\includegraphics[width=#4]{{video/#2_#3}.jpg}%
	};%
	\node[at=(A.center)] {%
		\ifcsname SlidesDistr\endcsname%
			\href{https://youtu.be/#3}{\includegraphics[width=2cm]{../assets/play_button.pdf}}%
		\else%
			\href{video/#2_#3.#1}{\includegraphics[width=2cm]{../assets/play_button.pdf}}%
		\fi%
	};%
	\end{tikzpicture}%
	\endgroup%
}

\newcommand{\backupbegin}{
	\newcounter{finalframe}
	\setcounter{finalframe}{\value{framenumber}}
	\setbeamertemplate{footline}{}
}

\newcommand{\backupend}{
	\setcounter{framenumber}{\value{finalframe}}
}


\date{Dec 4, 2023}
\title{SYDE 556/750 \\ Simulating Neurobiological Systems \\ Lecture 13: Conclusion}

\begin{document}
	
\begin{frame}{}
	\vspace{0.5cm}
	\begin{columns}[c]
		\column{0.6\textwidth}
		\MakeTitle
		\column{0.4\textwidth}
		\includegraphics[width=\textwidth]{media/jean_baptiste_marc_bourgery_atlas_of_anatomy_human_brain.jpg}
	\end{columns}
\end{frame}

\begin{frame}{Goal of This Course}
	\begin{overlayarea}{\textwidth}{7cm}
		\vspace{0.25cm}
		\begin{center}
			\only<2->{\huge Building Large-Scale Brain Models\\[0.25cm]}
			\only<3->{\large Why?}
		\end{center}
		\begin{columns}[T]
			\column{0.33\textwidth}
			\centering
			\only<4->{\includegraphics[height=3cm]{media/chimp_brain_in_a_jar_small.jpg}\\Understand how Brains Work}
			\column{0.33\textwidth}
			\centering
			\only<5->{\includegraphics[height=3cm]{media/CPBR2015_11_small.jpg}\\Build Better AI Systems}
			\column{0.33\textwidth}
			\centering
			\only<6->{\includegraphics[height=3cm]{media/necka1abcd-2881432-large.png}\\Program Neuromorphic Hardware}
		\end{columns}
	\end{overlayarea}
	
	\ImageSources{%
		Left: \enquote{A chimpanzee brain at the Science Museum London}, from  
		\href{https://commons.wikimedia.org/wiki/File:Chimp_Brain_in_a_jar.jpg}{Wikimedia}.
		Centre: \enquote{Robot at a campus faire in São Paulo} from
		\href{https://commons.wikimedia.org/wiki/File:CPBR2015_-_11.jpg}{Wikimedia}.
		Right: The Braindrop Neuromorphic hardware system,
		from \enquote{Braindrop: A Mixed-Signal Neuromorphic Architecture With a Dynamical Systems-Based Programming Model}, Neckar et al., 2019.}
\end{frame}

\begin{frame}{Theoretical Neuroscience}
	\centering
	\includegraphics[height=7cm]{media/levels.pdf}
\end{frame}

\begin{frame}{The Brain -- Some Statistics}
	\begin{itemize}
		\item \textbf{Weight:}\\
		\SI{2}{\kilogram} (2\% of the body weight)
		\item \textbf{Power consumption:}\\
		\SI{20}{\watt} (25\% of the body's total power consumption)
		\item \textbf{Surface area:}\\
		\SIrange{1500}{2000}{\centi\metre\squared} (roughly four A4/letter pages of paper)
		\item \textbf{Number of neurons:}\\
		\SI{100} billion ($10^{11}$, \SI{150000}{\per\milli\metre\squared})
		\item \textbf{Number of synapses:}\\
		100 trillion ($10^{14}$, about $1000$ per neuron)
	\end{itemize}	
\end{frame}


\begin{frame}{Neuromorphic Hardware}
	\centering
	\hl{Goal:} Brain-inspired hardware; lower power consumption; stream processing\\[0.25cm]
	\begin{overlayarea}{\textwidth}{6cm}
	\begin{columns}[t]
		\column{0.5\textwidth}
		\centering
		\only<2->{\hl{Digital}\\[0.5cm]
		\begin{itemize}
			\setlength{\itemsep}{0.25cm}
			\item<3-> Specialised digital hardware for simulating spiking neural networks
			\item<3-> Trivial weight-spike multiplication
			\item<3-> Often asynchronous (no central clock)
			\item[\OPlus]<5-> Deterministic
			\item[\OMeh]<5-> Higher power consumption than analogue
		\end{itemize}}
		\column{0.5\textwidth}
		\centering
		\only<2->{\hl{Analogue/Mixed Signal}\\[0.5cm]
		\begin{itemize}
			\setlength{\itemsep}{0.25cm}
			\item<4-> Neuron models in analogue hardware (capacitors, resistors, ...)
			\item<4-> Digital interconnect and programming (weights, neuron parameters)
			\item[\OMinus]<6-> Not deterministic
			\item[\OMinus]<6-> Hard to program
			\item[\OPlus]<6-> Very low power consumption
		\end{itemize}}
	\end{columns}
	\end{overlayarea}
\end{frame}

\begin{frame}{Neuromorphic Hardware -- SpiNNaker}
	\centering
	\begin{columns}
		\column{0.4\textwidth}
		\includegraphics[width=\textwidth]{media/spinnaker_48board.jpg}
		\column{0.6\textwidth}
		\begin{itemize}
			\setlength{\itemsep}{0.25cm}
			\item Manchester/Dresden collaboration; HBP
			\item Fully digital
			\item 18 ARM968 processors @ \SI{180}{\mega\hertz} per chip
			\item 1000 current-based LIF neurons per core
			\item Toroidal, asynchronous interconnect mesh
			\item Up to $\approx 10^{9}$ neurons in one system
			\item[\OPlus]<2-> Easy to program, (outdated) Nengo interface
			\item[\OPlus]<2-> Public access via HBP
			\item[\OMeh]<2-> Not very power efficient (version from 2013)
			\item[\OMinus]<2-> High setup times
		\end{itemize}
	\end{columns}
\end{frame}

\begin{frame}{Neuromorphic Hardware -- Loihi}
	\centering
	\begin{columns}
		\column{0.4\textwidth}
		\includegraphics[width=\textwidth]{media/loihi.jpg}\\
		\ImageSources{Intel Marketing Material}
		\column{0.6\textwidth}
		\begin{itemize}
			\setlength{\itemsep}{0.25cm}
			\item Developed by Intel
			\item Digital, fully asynchronous architecture
			\item Circuits accelerating individual spiking neurons
			\item<2->[\OPlus] Extremely low power consumption
			\item<2->[\OPlus] Nengo Interface
			\item<2->[\OMinus] Proprietary/no low level programming without signing an NDA
		\end{itemize}
	\end{columns}
\end{frame}

\begin{frame}{Neuromorphic Hardware -- BrainDrop}
	\centering
	\begin{columns}
		\column{0.4\textwidth}
		\includegraphics[width=\textwidth]{media/necka1abcd-2881432-large_2.jpg}\\[0.5cm]
		\ImageSources{Braindrop: A Mixed-Signal
Neuromorphic Architecture
With a Dynamical
Systems-Based
Programming Model, Neckar et al.~2019}
		\column{0.6\textwidth}
		\begin{itemize}
			\setlength{\itemsep}{0.25cm}
			\item Research project at Stanford
			\item Mixed signal, analogue neurons, synapse arrays
			\item Exploits process noise for diverse neural tuning
			\item Optimized for NEF networks
			\item<2->[\OPlus] Extremely low power consumption
			\item<2->[\OPlus] Nengo Interface
			\item<2->[\OMeh] Availability/Documentation?
			\item<2->[\OMinus] Small networks only
		\end{itemize}
	\end{columns}
\end{frame}

\begin{frame}{Neuromorphic Hardware -- BrainScaleS}
	\vspace{0.45cm}
	\centering
	\includegraphics[width=0.33\textwidth]{media/brain_scales.png}\hspace{0.5cm}%
	\includegraphics[width=0.33\textwidth]{media/brainscales_system1_small.jpg}\\[0.5cm]
	\begin{columns}
		\column{0.5\textwidth}
		\begin{itemize}
			\setlength{\itemsep}{0.25cm}
			\item Research Project in Heidelberg; HBP
			\item Mixed signal; above realtime
			\item Wafer-scale system; 384 x 256 neurons
		\end{itemize}
		\column{0.5\textwidth}
		\begin{itemize}
			\setlength{\itemsep}{0.25cm}
			\item<2->[\OPlus] Low power consumption
			\item<2->[\OPlus] Public access via HBP
			\item<2->[\OMinus] Complex; relatively low precision
		\end{itemize}
	\end{columns}
	\vspace{1cm}
	\ImageSources{Electronic Visions Group; Kirchhoff Institute for Physics; BrainScales/HBP Neuromorphics Project}
\end{frame}

\begin{frame}{Review: What Did We Learn? (I)}
	\begin{itemize}
		\item \textbf{The Neural Engineering Framework}\\[0.25cm]
		\begin{itemize}
			\setlength{\itemsep}{0.25cm}
			\item Theory for theoretical neuroscience (\hl{bridging laws})
			\item \emph{Principle 1:}\\Populations of neurons represent values $\vec x$
			\item \emph{Principle 2:}\\Connections compute functions $f$
			\item \emph{Principle 3:}\\Values are states in a dynamical system
			\item<2-> Model of \emph{how} the brain computes. \only<3->{Is it wrong?} \only<4->{Of course!} \only<5->{But hopefully useful!}
		\end{itemize}
	\end{itemize}
\end{frame}

\begin{frame}{Review: What Did We Learn? (II)}
	\begin{itemize}
		\item \textbf{Cognitive Architectures}\\[0.25cm]
		\begin{itemize}
			\setlength{\itemsep}{0.25cm}
			\item \emph{Jackendoff's Challenges:}\\
			How to explain language in neural networks?
			\item \emph{Vector Symbolic Algebras:}\\
			Compressing symbolic information into vectors,\\circular convolution, word embeddings
			\item \emph{Semantic Pointer Architecture:}\\
			Combination of four ideas -- \emph{What} the brain computes.
			\begin{itemize}
				\setlength{\itemsep}{0.125cm}
				\item VSAs
				\item NEF
				\item Deep Semantics: compression, decompression
				\item Architecture: Basal Ganglia/Thalamus/Cortex Loop
			\end{itemize}
		\end{itemize}
	\end{itemize}
\end{frame}


\begin{frame}{Review: What Did We Learn? (III)}
	\begin{itemize}
		\item \textbf{Methods \& Techniques}\\[0.25cm]
		\begin{itemize}
			\setlength{\itemsep}{0.25cm}
			\item \emph{The Delay Network:}\\
			Efficiently compress past history into a vector; optimal recurrent update rule
      \item \emph{Spatial Semantic Pointers:}\\
			Bio-inspired representation of continuous spaces; e.g., maps, probabilities
      \item \emph{Machine Learning:}\\
			Unsupervised, supervised learning, gradient descent,\\least squares, nonnegative least squares, delta learning rule (PES)
			\item \emph{Signal Processing:}\\
			Fourier, Laplace transformation; computing optimal filters
			\item \emph{Dimensionality Reduction:}\\
			PCA/SVD; function bases; Hebbian learning/Oja learning rule
		\end{itemize}
	\end{itemize}
\end{frame}


\begin{frame}{Review: What Did We Learn? (IV)}
	\begin{itemize}
		\item \textbf{\enquote{Meta-Level Skills}}\\[0.25cm]
		\begin{itemize}
			\setlength{\itemsep}{0.5cm}
			\item \emph{Solving Problems by Building a Signal Flow Graph:}\\
			Applications to Hardware design, differentiable computing
			\item \emph{Programming with Python/Numpy}
			\item \emph{Building Neural Networks using Nengo:}\\
			Can be applied to neuromorphic hardware (see above), cognitive modeling, machine learning
		\end{itemize}
	\end{itemize}
\end{frame}

\begin{frame}{Summary}
	\begin{itemize}
		\item \textbf{Party tricks \emph{or} How to impress your mom:}\\[0.25cm]
		\begin{itemize}
			\setlength{\itemsep}{0.5cm}
			\item \emph{Did you know you can't keep your eyes still in the dark?}\\
			They're controlled by the nuclei prepositus hypoglossi (NPH), part of the brainstem. I built one of those.
			\item \emph{Did you know people are building hardware that works like the brain?}\\
			It's called neuromorphic hardware, and uses spikes to communicate like the brain. I built some neural networks that can run on that hardware.
			\item \emph{I know how the world's largest brain model works.}\\
			In fact, I built some of the parts. Let me tell you about the working memory task.
		\end{itemize}
	\end{itemize}
\end{frame}

\begin{frame}{}
	\centering
	{\Huge Thank You!}
\end{frame}

\end{document}
