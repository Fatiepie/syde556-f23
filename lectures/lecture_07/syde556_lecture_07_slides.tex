% !TeX spellcheck = en_GB
\ifcsname SlidesDistr\endcsname%
\documentclass[handout,aspectratio=169]{beamer}
\else%
\documentclass[aspectratio=169]{beamer}
\fi%
\usepackage{fontspec}
\usepackage[T1]{fontenc}
\usepackage{amsmath}
\usepackage{amsfonts}
\usepackage{amssymb}
\usepackage{graphicx}
\usepackage{csquotes}
\usepackage{booktabs}
\usepackage{multicol}
\usepackage{enumerate}
\usepackage{microtype}
\usepackage[labelfont=bf,font={small}]{caption}
\usepackage{hyperref}
\usepackage{booktabs}
\usepackage{subcaption}
\usepackage{fancyhdr}
\usepackage{pdfpages}
\usepackage{siunitx}
\usepackage{tikz}
\usepackage{mdframed}

\defaultfontfeatures{Mapping=tex-text}
\newfontfamily\symbolfont{Symbola}
\newfontfamily\quotefont{GentiumPlus}

\usepackage[sorting=none]{biblatex}
\addbibresource{../bibliography.bib}

\author{Chris Eliasmith}


\renewcommand{\vec}[1]{{\mathbf{#1}}}
\newcommand{\mat}[1]{{\mathbf{#1}}}
\newcommand{\T}{\ensuremath{\mathsf{T}}}
\renewcommand{\epsilon}{\varepsilon}
\renewcommand{\phi}{\varphi}

% Tango color palette
\definecolor{butter1}{HTML}{FCE94F}
\definecolor{butter2}{HTML}{EDD400}
\definecolor{butter3}{HTML}{C4A000}
\definecolor{orange1}{HTML}{FCAF3E}
\definecolor{orange2}{HTML}{F57900}
\definecolor{orange3}{HTML}{CE5C00}
\definecolor{chocolate1}{HTML}{E9B96E}
\definecolor{chocolate2}{HTML}{C17D11}
\definecolor{chocolate3}{HTML}{8F5902}
\definecolor{chameleon1}{HTML}{8AE234}
\definecolor{chameleon2}{HTML}{73D216}
\definecolor{chameleon3}{HTML}{4E9A06}
\definecolor{skyblue1}{HTML}{729FCF}
\definecolor{skyblue2}{HTML}{3465A4}
\definecolor{skyblue3}{HTML}{204A87}
\definecolor{plum1}{HTML}{AD7FA8}
\definecolor{plum2}{HTML}{75507B}
\definecolor{plum3}{HTML}{5C3566}
\definecolor{scarletred1}{HTML}{EF2929}
\definecolor{scarletred2}{HTML}{CC0000}
\definecolor{scarletred3}{HTML}{A40000}
\definecolor{aluminium1}{HTML}{EEEEEC}
\definecolor{aluminium2}{HTML}{D3D7CF}
\definecolor{aluminium3}{HTML}{BABDB6}
\definecolor{aluminium4}{HTML}{888A85}
\definecolor{aluminium5}{HTML}{555753}
\definecolor{aluminium6}{HTML}{2E3436}

\definecolor{violet}{HTML}{AA305C}
\definecolor{uwyellow}{HTML}{FDD433}
\definecolor{background}{HTML}{F9F9F6}
\definecolor{text}{HTML}{000000}

\definecolor{uweng1}{HTML}{D1B2EE}
\definecolor{uweng2}{HTML}{BF33DE}
\definecolor{uweng3}{HTML}{8001B3}
\definecolor{uweng4}{HTML}{56048A}

\setbeamercolor{title}{fg=violet}
\setbeamercolor{frametitle}{fg=black}
\setbeamercolor{structure}{fg=aluminium5}
\setbeamercolor{normal text}{fg=text}

\setbeamertemplate{navigation symbols}{}
\setbeamertemplate{footline}[frame number]

\hypersetup{%
	colorlinks=false,% hyperlinks will be black
	urlbordercolor=aluminium4,% hyperlink borders will be red
	pdfborderstyle={/S/U/W 0.5}% border style will be underline of width 1pt
}

\makeatletter
\newcommand{\superimpose}[2]{%
	{\ooalign{{#1}\hidewidth\cr{#2}\hidewidth\cr}}}
\makeatother
\newcommand{\SolidCircle}[2]{\superimpose{\color{#1}\symbolfont ⬤}{\textbf{\color{white}#2}}\hspace{1em}}
\newcommand{\OPlus}{\SolidCircle{chameleon3}{\kern0.75pt+}}
\newcommand{\OMeh}{\SolidCircle{uwyellow}{~}}
\newcommand{\OMinus}{\SolidCircle{scarletred3}{\kern2.25pt--}}

\newcommand{\hl}[1]{\colorbox{uwyellow}{{\color{black}\textbf{#1}}}}

\newcommand{\ImageSources}[1]{%
	\begin{columns}%
		\column{1.1\textwidth}%
		\raggedright%
		\tiny\color{aluminium4}%
		\setlength\lineskip{1em}%
		\textbf{Image Sources.}	{#1}%
	\end{columns}}

\newcommand{\ColorRect}[3]{{\color{#1}\rule{#2}{#3}}}
\setbeamertemplate{headline}{\ColorRect{black}{\textwidth}{4pt}\newline\ColorRect{uweng1}{0.25\textwidth}{4pt}\ColorRect{uweng2}{0.25\textwidth}{4pt}\ColorRect{uweng3}{0.25\textwidth}{4pt}\ColorRect{uweng4}{0.25\textwidth}{4pt}}

\newcommand{\MakeTitle}{%
	\vspace{0.5cm}%
	{\textbf{\inserttitle}}\\[0.5cm]%
	\insertauthor\\[0.5cm]%
	\insertdate\\%
	\begin{itemize}
		\item Slide design: Andreas Stöckel
		\item Content: Terry Stewart, Andreas Stöckel, Chris Eliasmith
	\end{itemize}
 	\includegraphics[width=7cm]{../assets/uwlogo_eng.pdf}%
}

\newcommand{\handwritingframe}{%
	\begin{frame}
		\begin{columns}
			\column{\paperwidth}
			\includegraphics{../assets/handwriting_lines.pdf}
		\end{columns}
	\end{frame}	
}

\newcommand{\imageframe}[1]{%
	\setbeamertemplate{navigation symbols}{}%
	\begin{frame}[plain,noframenumbering]%
		\begin{tikzpicture}[remember picture,overlay]%
		\node[at=(current page.center)] {%
			\includegraphics[width=\paperwidth]{#1}%
		};%
		\end{tikzpicture}%
	\end{frame}%
}

\newcommand{\videoframe}[3][mp4]{%
	\begin{frame}[plain,noframenumbering]%
		\hypersetup{%
			pdfborderstyle={/S/U/W 0}% border style will be underline of width 1pt
		}%
		\begin{tikzpicture}[remember picture,overlay]%
		\node[at=(current page.center)] {%
			\includegraphics[width=\paperwidth]{{video/#2_#3}.jpg}%
		};%
		\node[at=(current page.center)] {%
			\ifcsname SlidesDistr\endcsname%
				\href{https://youtu.be/#3}{\includegraphics[width=2cm]{../assets/play_button.pdf}}%
			\else%
				\href{video/#2_#3.#1}{\includegraphics[width=2cm]{../assets/play_button.pdf}}%
			\fi%
		};%
		\end{tikzpicture}%
	\end{frame}%
}

\newcommand{\includevideo}[4][mp4]{%
	\begingroup%
	\hypersetup{%
		pdfborderstyle={/S/U/W 0}% border style will be underline of width 1pt
	}%
	\begin{tikzpicture}%
	\node (A) {%
		\includegraphics[width=#4]{{video/#2_#3}.jpg}%
	};%
	\node[at=(A.center)] {%
		\ifcsname SlidesDistr\endcsname%
			\href{https://youtu.be/#3}{\includegraphics[width=2cm]{../assets/play_button.pdf}}%
		\else%
			\href{video/#2_#3.#1}{\includegraphics[width=2cm]{../assets/play_button.pdf}}%
		\fi%
	};%
	\end{tikzpicture}%
	\endgroup%
}

\newcommand{\backupbegin}{
	\newcounter{finalframe}
	\setcounter{finalframe}{\value{framenumber}}
	\setbeamertemplate{footline}{}
}

\newcommand{\backupend}{
	\setcounter{framenumber}{\value{finalframe}}
}


\date{October 23, 2023}
\title{SYDE 556/750 \\ Simulating Neurobiological Systems \\ Lecture 7: Temporal Basis Functions}

\begin{document}
	
	\begin{frame}{}
		\vspace{0.5cm}
		\begin{columns}[c]
			\column{0.6\textwidth}
			\MakeTitle
			\column{0.4\textwidth}
			\includegraphics[width=\textwidth]{media/a22_2007_06_einsteinturm_sonnenuhr_p6070062.jpg}
		\end{columns}
	\end{frame}

	\begin{frame}{Representing Stimulus Histories}
		\centering%
		\includegraphics[width=\textwidth]{media/delay_network_decoding.pdf}%
	\end{frame}

	\begin{frame}{Representing Functions: Sampling}
		\centering%
		\includegraphics<1>[width=\textwidth]{media/fun_1_q4_1.pdf}%
		\includegraphics<2>[width=\textwidth]{media/fun_1_q4_2.pdf}%
		\includegraphics<3>[width=\textwidth]{media/fun_1_q4_3.pdf}%
		\includegraphics<4>[width=\textwidth]{media/fun_1_q4_4.pdf}%
		\includegraphics<5>[width=\textwidth]{media/fun_1_q4_5.pdf}%
		\includegraphics<6>[width=\textwidth]{media/fun_1_q4_6.pdf}%
%		\includegraphics<7>[width=\textwidth]{media/fun_1_q6_1.pdf}%
		\includegraphics<7>[width=\textwidth]{media/fun_1_q6_2.pdf}%
		\includegraphics<8>[width=\textwidth]{media/fun_1_q6_3.pdf}%
		\includegraphics<9>[width=\textwidth]{media/fun_1_q6_4.pdf}%
		\includegraphics<10>[width=\textwidth]{media/fun_1_q6_5.pdf}%
		\includegraphics<11>[width=\textwidth]{media/fun_1_q6_6.pdf}%
		\includegraphics<12>[width=\textwidth]{media/fun_2_q6_1.pdf}%
		\includegraphics<13>[width=\textwidth]{media/fun_2_q6_2.pdf}%
		\includegraphics<14>[width=\textwidth]{media/fun_2_q6_3.pdf}%
		\includegraphics<15>[width=\textwidth]{media/fun_2_q6_4.pdf}%
		\includegraphics<16>[width=\textwidth]{media/fun_2_q6_5.pdf}%
		\includegraphics<17>[width=\textwidth]{media/fun_2_q6_6.pdf}%
		\includegraphics<18>[width=\textwidth]{media/fun_2_q6_small.pdf}%
	\end{frame}

	\begin{frame}{Representing Functions: Fourier Basis}
		\includegraphics<1>[width=\textwidth]{media/fourier_q1.pdf}%
		\includegraphics<2>[width=\textwidth]{media/fourier_q2.pdf}%
		\includegraphics<3>[width=\textwidth]{media/fourier_q3.pdf}%
		\includegraphics<4>[width=\textwidth]{media/fourier_q4.pdf}%
		\includegraphics<5>[width=\textwidth]{media/fourier_q5.pdf}%
		\includegraphics<6>[width=\textwidth]{media/fourier_q6.pdf}%
	\end{frame}

	\begin{frame}{Representing Functions: Cosine Basis}
		\includegraphics<1>[width=\textwidth]{media/cosine_q1.pdf}%
		\includegraphics<2>[width=\textwidth]{media/cosine_q2.pdf}%
		\includegraphics<3>[width=\textwidth]{media/cosine_q3.pdf}%
		\includegraphics<4>[width=\textwidth]{media/cosine_q4.pdf}%
		\includegraphics<5>[width=\textwidth]{media/cosine_q5.pdf}%
		\includegraphics<6>[width=\textwidth]{media/cosine_q6.pdf}%
	\end{frame}

	\begin{frame}{Representing Functions: ReLU Basis}
		\includegraphics<1>[width=\textwidth]{media/neural_q1.pdf}%
		\includegraphics<2>[width=\textwidth]{media/neural_q2.pdf}%
		\includegraphics<3>[width=\textwidth]{media/neural_q3.pdf}%
		\includegraphics<4>[width=\textwidth]{media/neural_q4.pdf}%
		\includegraphics<5>[width=\textwidth]{media/neural_q5.pdf}%
		\includegraphics<6>[width=\textwidth]{media/neural_q6.pdf}%
	\end{frame}

	\begin{frame}{Representing Functions: Legendre Basis}
		\includegraphics<1>[width=\textwidth]{media/legendre_q1.pdf}%
		\includegraphics<2>[width=\textwidth]{media/legendre_q2.pdf}%
		\includegraphics<3>[width=\textwidth]{media/legendre_q3.pdf}%
		\includegraphics<4>[width=\textwidth]{media/legendre_q4.pdf}%
		\includegraphics<5>[width=\textwidth]{media/legendre_q5.pdf}%
		\includegraphics<6>[width=\textwidth]{media/legendre_q6.pdf}%
	\end{frame}

	\begin{frame}{Implementing the Delay Network}
		\centering%
		\begin{columns}
			\column{0.5\textwidth}
			\centering
			\includegraphics{media/lti_integrator_vs_neural_a.pdf}%
			\column{0.5\textwidth}
			\centering
			\includegraphics{media/lti_integrator_vs_neural_b.pdf}%
		\end{columns}
		\begin{align*}
			\theta \mat{A} &= a_{ij} \in \mathbb{R}^{q \times q} \,, & \kern-0.5em a_{ij} &= \begin{cases}
			(2i + 1)(-1) & \kern-0.5em i < j \,,\\
			(2i + 1)(-1)^{i - j + 1} & \kern-0.5em i \geq j
			\end{cases}\\
			\theta \mat{B} &= b_i \in \mathbb{R}^{q} \,, & \kern-0.5em b_i &= (2i + 1)(-1)^i \\
			\mat A' &= \tau \mat A + \mat I & \mat B' &= \tau \mat B
		\end{align*}
	\end{frame}

	\begin{frame}{Delay Network: Step Function}
		\centering%
		\includegraphics<1>[width=\textwidth]{media/delay_network_step_q1.pdf}%
		\includegraphics<2>[width=\textwidth]{media/delay_network_step_q2.pdf}%
		\includegraphics<3>[width=\textwidth]{media/delay_network_step_q3.pdf}%
		\includegraphics<4>[width=\textwidth]{media/delay_network_step_q4.pdf}%
		\includegraphics<5>[width=\textwidth]{media/delay_network_step_q5.pdf}%
		\includegraphics<6>[width=\textwidth]{media/delay_network_step_q6.pdf}%
		\includegraphics<7>[width=\textwidth]{media/delay_network_step_q11.pdf}%
	\end{frame}

	\begin{frame}{Delay Network: Windowed Sine Function}
		\centering%
		\includegraphics<1>[width=\textwidth]{media/delay_network_sine_i0_q6.pdf}%
		\includegraphics<2>[width=\textwidth]{media/delay_network_sine_i1_q6.pdf}%
		\includegraphics<3>[width=\textwidth]{media/delay_network_sine_i2_q6.pdf}%
	\end{frame}

	\backupbegin

	\begin{frame}[noframenumbering]{Image sources}
		\small
		\textbf{Title slide}\\Infrared Photograph of a Sundial Near the Einstein Tower in Potsdam, Germany\\Author: DrNRNowaczyk, 2007.\\From \href{https://commons.wikimedia.org/wiki/File:A22_2007-06_Einsteinturm_Sonnenuhr_p6070062.jpg}{Wikimedia}.
	\end{frame}


	\backupend
	
\end{document}
